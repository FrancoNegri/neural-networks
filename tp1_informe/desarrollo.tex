\subsection{Preprocesamiento De Los Datos} 
Con el objetivo de darle mayor estabilidad matematica a los datos y quitar algo del ruido que puedan tener, realizamos un preprocesamiento de los mismos. Para ello, primero sacamos los utliers dentro de nuestras muestras y luego procedemos a normalizarlos con media 0 y varianza 1.

\subsection{Implementacion Del algoritmo} 

El algoritmo utilizado es un back propagation con learning rate adaptativo y momentum.
%completar

\subsection{Performance Diagnostico de Cancer} 

Tras repetidas experimentaciones cambiando la cantidad de nueronas, learning rate, cantidad de capas, no pudo encontrarse una solucion razonable para el problema. La solucion del algoritmo convergió a valores poco adecuados y que no aportaban ningun tipo de informacion util. (Cancer = -1, No cancer = 1, prediccion = 0.0) por lo que consideramos que los atributos dados no son adecuados para la solucion de este problema
%completar
%Graficos

\subsection{Performance Diagnostico de Cancer} 

%falta implementar