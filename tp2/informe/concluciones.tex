PCA: Podemos observar que a medida que aumentamos la cantidad de épocas los datos tienden a esparcirse mas, y esto claramente es debido a que el algoritmo lo que hace es maximizar la varianza. Es decir, en otros términos lo que trata de hacer es lo siguiente:
Tenemos 3 ejes en los cuales podemos representar nuestros datos. Inicialmente todos los puntos estan muy pegados los unos a los otros. Es como si tuviesemos una cámara fotográfica y sacaramos una foto en cierto ángulo viendo todos los puntos unos encima de los otros. A medida que aumentan las iteraciones lo que el algoritmo trata de hacer es cambiar el ángulo de nuestra cámara para poder ver a cada uno de los puntos de manera mas separada posible de sus aledaños para poder en una misma captura, verlos a todos sin que se encimen. Por eso cuando el algoritmo converge, podemos observar a los puntos esparcidos en el espacio, obteniendo la mayor cantidad de información (ya que por ejemplo, si estan todos pegoteados, no tengo forma de separarlos mediante planos por ejemplo para clasificacion, ahora bien si los trato de representar en un sistema de componentes principales quizá estos se separen lo suficiente que me permitan separarlos).

SOM: Podemos ver que claramente en un 25\% de entrenamiento, el algoritmo ya logro su ordenamiento. Lo que resta (es decir las iteraciones restantes) son simplemente para la convergencia del mismo. Esto es fundamentado en el libro de Haykin donde se establece que alrededor de 1000 iteraciones hacen falta para la fase de auto organización de los datos. Luego la fase de convergencia depende fuertemente de la dimensionalidad de nuestro input, pudiendo necesitar una cantidad muy alta de iteraciones. Concuimos a través de la experimentación que el algoritmo pudo hacer una buena clusterización de los datos en distintos grupos bien diferenciados.